% настройка кодировки, шрифтов и русского языка
\usepackage{fontspec}
\usepackage{polyglossia}

% рабочие ссылки в документе
\usepackage{hyperref}

% графика
\usepackage{graphicx}
\usepackage{tikz}

% поворот страницы
\usepackage{pdflscape}

% качественные листинги кода
\usepackage{minted}
\usepackage{listings}
\usepackage{lstfiracode}

% отключение копирования номеров строк из листинга, работает не во всех просмотрщиках (в Adobe Reader работает)
\usepackage{accsupp}
\newcommand\emptyaccsupp[1]{\BeginAccSupp{ActualText={}}#1\EndAccSupp{}}
\let\theHFancyVerbLine\theFancyVerbLine
\def\theFancyVerbLine{\rmfamily\tiny\emptyaccsupp{\arabic{FancyVerbLine}}}

% библиография
\bibliographystyle{templates/gost-numeric.bbx}
\usepackage{csquotes}
\usepackage[parentracker=true,backend=biber,hyperref=true,bibencoding=utf8,style=numeric-comp,language=auto,autolang=other,citestyle=gost-numeric,defernumbers=true,bibstyle=gost-numeric,sorting=ntvy]{biblatex}

% установка полей
\usepackage{geometry}

% нумерация картинок по секциям
\usepackage{chngcntr}

% дополнительные команды для таблиц
\usepackage{booktabs}

% для заголовков
\usepackage{caption}
\usepackage{titlesec}
\usepackage[dotinlabels]{titletoc}

% разное для математики
\usepackage{amsmath, amsfonts, amssymb, amsthm, mathtools}

% водяной знак на документе, см. main.tex
\usepackage[printwatermark]{xwatermark}

% Set fallback font
\newcommand*{\fallbackfont}{\fontfamily{Sans Serif}\selectfont}

% Fallback in different cases (юзаем другие шрифты если все плохо)
\usepackage{unicode-math}
\usepackage{newunicodechar}
\newunicodechar{№}{{\fallbackfont №}}

